\documentclass{article}
\usepackage{amsmath}
\title{Making an LP Matrix}
\author{Gary Morales}
\date{May 21, 2013}
\begin{document}
\maketitle
\section{Introduction}
As a reminder I am working in a group with Andrea, Brian, and Melissa.
We are working on a program that acts as a tutorial for the simplex algorithm.

Section \ref{goals} lists what we hope our program will acomplish

Section \ref{mywork} describes what part of the project I have been working on.
\subsection{Goals}\label{goals}
Here is a list of what our program will do
\begin{enumerate}
    \item Solve linear programs
    \item Output solution as vector of values
    \item Output final tableau
    \item Display tableau after each step of simplex algorithm
    \item Display graph of feasible solutions
\end{enumerate}
\subsection{My part}\label{mywork}
Since I am working in a group I have only worked on part of the project.
In particular I am working on the input of our program.

Given a linear progam of the form:

\begin{array}{lrrrr}
    \textrm{Max}  & 2x_1 & + 3x_2 & + 4x_3 & \\
    \textrm{s.t.} & 3x_1 & + 2x_2 & - x_3  & \leq 5\\
                  & 2x_1 & - 2x_2 &        & \leq 4\\
                  &      &        &   x_3  & \leq 5
\end{array}

We want to take the input as a list of expressions, representing the restrictions, and single
expression to represent the objective function

\section{What I've done}

So far, what I have been able to do is make a function that takes the desired input and converts it into a matrix in the form a list of lists, with each inner list representing a row of the matrix. This matrix is supposed to represent the initial tableau and will be what we use to run the algorithm.

\subsection{Results}
The resulting matrix for the linear programming given in \ref{mywork} looks like:

\left(\begin{array}{rrrrrrr}
    3  & -1 &  2 & 1 & 0 & 0 & 5 \\
    2  &  0 & -2 & 0 & 1 & 0 & 4 \\
    0  &  1 &  0 & 0 & 0 & 1 & 5 \\
    -2 & -4 & -3 & 0 & 0 & 0 & 0
\end{array}\right)

One thing you will notice is tha the second column corresponds to the coefficients associated with the variable $x_3$ and not $x_2$ as would seem more natural. I do not know why but set([x1,x2,x3]) gives \verb|set([x1,|\phantom{\verb!x!}\verb|x3,|\phantom{\verb!x!}\verb|x2])|
as the output as opposed to this order: $x_1,x_2,x_3$. The union function does a similar thing as well. I will probably end up making this into a class and having the order as a field.
\end{document}